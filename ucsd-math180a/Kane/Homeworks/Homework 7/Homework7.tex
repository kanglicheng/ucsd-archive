\documentclass{article}
\usepackage{amsthm, amssymb, amsmath,verbatim}
\usepackage[margin=1in]{geometry}
\usepackage{enumerate}
\usepackage{graphicx}

\newcommand{\R}{\mathbb{R}}
\newcommand{\C}{\mathbb{C}}
\newcommand{\Z}{\mathbb{Z}}
\newcommand{\F}{\mathbb{F}}
\newcommand{\N}{\mathbb{N}}



\newtheorem*{claim}{Claim}
\newtheorem{ques}{Question}


\title{Math 184A Homework 7}
\date{Spring 2018}

\begin{document}

\maketitle

This homework is due on gradescope by Friday June 8th at 11:59pm. Remember to justify your work even if the problem does not explicitly say so. Writing your solutions in \LaTeX is recommend though not required.

\begin{ques}[Avoidance Bounds, 20 points]
From the book we know that $S_n(1432)\leq 9^n$. Find a constant $C$ so that $S_n(321456987)\leq C^n$ for all $n$.
\end{ques}

\begin{ques}[Hill Avoidance, 40 points]
Let a $k$-hill in a permutation be a subsequence of $2k-1$ of the entries the first $k$ of which are in increasing order and the last $k$ of which are in decreasing order. Note that a $k$-hill is not a single pattern. For example, a $2$-hill is either an instance of the pattern $132$ or an instance of the pattern $231$.
\begin{enumerate}[(a)]
\item Show that the number of permutations of $[n]$ with no $2$-hill is $2^{n-1}$. [15 points]
\item Show that the number of permutations of $[n]$ with no $k$-hill is at most $(4(k-1)^2)^n$. [Hint: try to find a decreasing sequence among elements that are the largest of a $k$-term increasing subsequence.] [25 points]
\end{enumerate}
\end{ques}

\begin{ques}[Marriage Lemma, 40 points]
The Marriage Lemma states that if you are given two sets $S$ and $T$ of size $n$ and a set $E$ of pairs of one element of each set, then there is a matching between $S$ and $T$ (namely a set of $n$ pairs from $E$ using each element of $S$ and each element of $T$ exactly once) unless there is some subset $S'\subset S$ so that the total number of elements of $T$ that pair with some element of $S'$ is less than $|S'|$.

Prove the Marriage Lemma using Dilworth's Theorem.
\end{ques}

\begin{ques}[Extra credit, 1 point]
Free point!
\end{ques}

\end{document} 