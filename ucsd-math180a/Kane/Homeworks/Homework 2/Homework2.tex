\documentclass{article}
\usepackage{amsthm, amssymb, amsmath,verbatim}
\usepackage[margin=1in]{geometry}
\usepackage{enumerate}
\usepackage{graphicx}

\newcommand{\R}{\mathbb{R}}
\newcommand{\C}{\mathbb{C}}
\newcommand{\Z}{\mathbb{Z}}
\newcommand{\F}{\mathbb{F}}
\newcommand{\N}{\mathbb{N}}



\newtheorem*{claim}{Claim}
\newtheorem{ques}{Question}


\title{Math 184A Homework 2}
\date{Spring 2018}

\begin{document}

\maketitle

This homework is due on gradescope by Friday April 20th at 11:59pm. Remember to justify your work even if the problem does not explicitly say so. Writing your solutions in \LaTeX is recommend though not required.

\begin{ques}[Partition Recurrence, 15 points]
We didn't mention in class any method to compute partition numbers, but there is a relatively simple recurrence relation that can be used for them. Prove that for all $n\geq k \geq 1$ that
$$
p_k(n) = \sum_{i=0}^k p_i(n-k).
$$
\end{ques}

\begin{ques}[Partitions with Sequential Part Sizes, 15 points]
Show that the number of partitions of $n$ into parts of distinct sizes is the same as the number of partitions of $n$ so that the adjacent parts have sizes differing by at most $1$ (so in particular $a_i \geq a_{i+1} \geq a_i-1$) and the smallest part has size 1.
\end{ques}

\begin{ques}[Compositions and Fibonacci Numbers, 30 points]
.

\begin{enumerate}[(a)]
\item Show that the number of compositions of $n$ into odd parts is the same as the number of compositions of $n-1$ into parts of size $1$ and $2$ for all $n\geq 1$. [15 points]
\item Define the Fibonacci numbers by the recurrence relation $F_1=F_2=1$ and $F_n=F_{n-1}+F_{n-2}$ for all $n\geq 3$. Show that the number of compositions of $n$ into odd parts is $F_n$ for all $n\geq 0$. [15 points]
\end{enumerate}
\end{ques}

\begin{ques}[Summation Polynomials, 40 points]
.

\begin{enumerate}[(a)]
\item Show that the number of compositions of $n$ into $k$ parts is the sum of $m$ going from $0$ to $n-1$ of the number of compositions of $m$ into $k-1$ parts. [10 points]
\item Show that for any $n$ and $k$ that
$$
\sum_{i=0}^n \binom{i}{k} = \binom{n+1}{k+1}.
$$
[10 points]
\item Recall that
$$
x^m = \sum_{k=0}^m k! S(m,k) \binom{x}{k}.
$$
We would like to come up with a formula for
$$
\sum_{i=0}^n i^m = P_m(n).
$$
In particular, we claim that for each $m$, we claim that $P_m(n)$ is a polynomial in $n$. For example,
$$
\sum_{i=0}^n i = \frac{n(n+1)}{2},
$$
so $P_1(n) = n(n+1)/2.$ Using the above formula and the result in part (b), give a formula for $P_m(n)$ in terms of Stirling numbers, and binomial coefficients. [20 points]
\end{enumerate}
\end{ques}

\begin{ques}[Extra credit, 1 point]
Approximately how much time did you spend on this homework?
\end{ques}

\end{document} 